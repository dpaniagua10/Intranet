\documentclass[10pt,a4paper]{article}
\usepackage[utf8]{inputenc}
\usepackage{amsmath}
\usepackage{amsfonts}
\usepackage{amssymb}
\usepackage{graphicx}
\title{Firewall Fortigate}
\begin{document}
\maketitle
	Fortinet nos permite descargar y utilizar la máquina virtual de forma gratuita durante dos semanas . Significa que no tiene que comprar nada para probar algunas funciones en su VM.
	
\section{Instalacion}

Lo primero que debe hacer es ir a support.fortinet.com y crear una cuenta gratuita.

Extraiga el archivo descargado y haga doble clic en el archivo de la máquina virtual.

Esto abrirá la ventana Importar máquina virtual donde deberá especificar la ubicación de la nueva máquina virtual:

\includegraphics[width=0.4\linewidth]{images/140584107-4dcd14f6-0702-4a78-adf2-0519110b8f25.png}

Ahora puede iniciar la VM y acceder a la consola CLI. El nombre de usuario predeterminado es admin y la contraseña está en blanco. En el primer inicio de sesión, se le pedirá que establezca la nueva contraseña. A continuación se muestra la captura de pantalla con el proceso de inicio de sesión inicial.


\includegraphics[width=0.6\linewidth]{images/140584153-7cd2e528-c4b7-464b-b0ff-e014fed00174.png}


De forma predeterminada, todos los adaptadores de red de la máquina virtual están en modo puente, esto significa que una de las interfaces obtendrá una dirección IP de su enrutador (servidor DHCP), que proporciona acceso a Internet a su computadora. Una vez que sepa la dirección IP de Fortigate-VM, puede iniciar sesión en su interfaz web. Si tuvo un error relacionado con la licencia, mencionado anteriormente


Este error tiene algo que ver con la sincronización horaria. La forma más fácil de hacer que funcione es simplemente realizar un restablecimiento de fábrica de Fortigate-VM. El comando se ejecuta el restablecimiento de fábrica 


Después del proceso de restablecimiento, deberá volver a iniciar sesión con las credenciales predeterminadas ( administrador y contraseña en blanco). Proporcione la nueva contraseña y busque la dirección IP de la interfaz como se describió anteriormente. Ahora podrá acceder a la GUI y comenzar a configurar el dispositivo.

\section{Topologia para el Firewall}

\includegraphics[width=0.6\linewidth]{images/140599214-35f962aa-1751-4672-9a99-bb45c0ace5a0.png}

\section{Politicas Firewall}

\section*{ruta estatica para navegacion de firewall y red lan}

se debe de configurar un gateway por el cual nuestro firewall perimitral tendra acceso a internet por medio de este a traves de politica brindaremos navegacion y acceso de red a los equipos conectados a la red Lan y DMZ.

\includegraphics[width=1\linewidth]{images/140585170-4c8d3b7e-f534-4518-a8e4-ced5c12009d1.png}

\section*{configuracion de interfases}

interfas WAN\\

se configuro la interfas Wan en el puerto 1 de nuestro firewall asignando una IP de nuestro segmento del router de casa, esto para utilizar el router de casa como puerta de salida del firewall.\\

\includegraphics[width=1\linewidth]{images/140597966-0836fddf-2b8e-4465-ad13-1f669c98f244.png}\\

interfas LAN\\

se configuro el puerto 2 de nuestro firewall como segmente LAN asignando la red 192.168.44.0/24 para los equipos windows server y windows endpoint los cuales pertenecen a nuestro segmente de red interna.\\

\includegraphics[width=0.8\linewidth]{images/140598083-7adacae0-0d75-4c40-830b-8658e4408aa8.png
}
\\

Interfas DMZ\\

se configuro el puerto 3 de nuestro firewall como segmente DMZ asignando la red 10.10.10.0/24 para los equipos Linux en los cuales se encuentran servicios criticos tales como base de datos y servidores FTP. Esta zona se encuentra aislada de la RED LAN y para acceder a ella se deben de solicitar permisos al administrador del FW para que cree politicas de acceso, indicando que equipos se van a conectar y protocolos a utilizar.\\


\includegraphics[width=0.8\linewidth]{images/140598403-cc47eec1-c449-4f8a-a06a-d8c519dc16ce.png}


\section*{
Creacion de objetos}

Según buenas practicas en equipos como firewall perimetral es recomendado crear objetos especificos para brindar accesos a los equipos a los diferentes destinos. Se crearon objetos como RED LAN, Servidor Ubuntu, RED DMZ entre otros.\\

\includegraphics[width=0.8\linewidth]{images/140598508-60acfb58-813f-4d9a-869c-c1f57a79f957.png}


\section*{Creacion politicas de acceso
}

Creacion politicas de acceso RED LAN\\

Se configuraron politicas de acceso a internet para los equipos conectados a la RED LAN, para esto fue necesario especificar las interfaces de entrada y salida así como que dispositivos tienen permisos y cuales no, en este tipo de reglas se pueden aplicar politicas de seguridad como perfiles UTM, en el cual se pueden configurar servicios y aplicaciones ya sea que se necesiten permitir o bloquear, en nuestro caso no fue posible activar este tipo de seguridad por temas de licenciamiento.\\


\includegraphics[width=0.8\linewidth]{images/140598562-11384857-6a40-453e-a491-f5928571caef.png}\\

Creacion politicas de acceso RED DMZ\\

Se configuraron politicas de acceso a internet para los equipos conectados a la RED DMZ, para esto fue necesario especificar las interfaces de entrada y salida así como que dispositivos tienen permisos y cuales no, esta politica se creó más cerrada por motivos de los servicios que se encuentran en la misma, cabe mencionar que esta zona no tiene acceso full a la zona LAN esto por motivos de seguridad.\\


\includegraphics[width=0.8\linewidth]{images/140598633-ad88fbd2-4146-4fe1-be7e-564d81d1e98f.png}\\


Comunicación red LAN con DMZ\\

Por politicas de seguridad la RED lan no debe de compartir todos los recursos con la RED DMZ, esto para evitar cualquier tipo de fuga de información o infección que se pueda dar en cualquier red, el firewall tiene como funcion brindar accesos o denegar permisos a los usuarios, para nuestro proyecto se creó una politica para comunicación entre RED LAN y DMZ únicamente permitiendo el servicio de mysql y ICMP, esto para mitigar cualquier riesgo o vulnerabilidad que utilize puertos inusuales.\\


\includegraphics[width=0.8\linewidth]{images/140598826-16fa46f6-4789-45cd-a39b-73e7292c4960.png}

\section*{
Monitorio de red}

Una de las muchas bondades de un firewall perimetral es la reporteria, por medio de esta nosotros podemos verificar tanto como vitalidad y accesos de los usuarios, en nuestro firewall configuramos un dashboard el cual utilizaremos para verificar el estado de las interfaces y el estado del firewall, monitoreando su performance como cpu, memoria ram y throughput de las interfaces configuradas.\\

\includegraphics[width=0.8\linewidth]{images/140598962-45b068e4-2cde-4e0c-9dc4-6034fe0c7740.png
}
\end{document}